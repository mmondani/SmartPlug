\chapter{Diseño e Implementación} % Main chapter title

\label{Chapter3} % Change X to a consecutive number; for referencing this chapter elsewhere, use \ref{ChapterX}
\definecolor{mygreen}{rgb}{0,0.6,0}
\definecolor{mygray}{rgb}{0.5,0.5,0.5}
\definecolor{mymauve}{rgb}{0.58,0,0.82}

%----------------------------------------------------------------------------------------

En este capítulo se explican los criterios utilizados en el desarrollo del prototipo y se justifican las decisiones de diseño, así como también se describe la implementación.

\section{Hardware}
\label{section:hardware}

En el contexto del presente trabajo, se desarrolló un prototipo funcional del producto final. Este prototipo presenta las mismas funciones que el equipo final pero no se ajusta a los lineamientos estéticos y mecánicos que si deberá cumplir en un futuro. Es por esto que el prototipo diseñado tiene dimensiones mucho más grandes que las que tendrá, lo cual facilitó las mediciones y pruebas que se debieron realizar para comprobar el correcto funcionamiento tanto del hardware como del firmware. 

\begin{figure}[!h]
	\centering
	\includegraphics[width=14cm]{./Figures/comercial.png}
\end{figure}

\section{Firmware}

El firmware embebido en el Smart Plug se escribió utilizando FreeOSEK como sistema operativo de tiempo real. 

\begin{figure}[!h]
	\centering
	\includegraphics[width=14cm]{./Figures/comercial.png}
\end{figure}


\section{Aplicación Android}
\label{section:app}

\begin{figure}[!h]
	\centering
	\includegraphics[width=14cm]{./Figures/comercial.png}
\end{figure}
